\documentclass[11pt]{article}
\everymath{\displaystyle}
\usepackage[utf8]{inputenc}  
\usepackage[T1]{fontenc}
\usepackage[francais]{babel}
\usepackage{amsmath,textcomp,amssymb,geometry,graphicx,enumerate}
\usepackage{algorithm} % Boxes/formatting around algorithms
\usepackage[noend]{algpseudocode} % Algorithms
\usepackage{hyperref}
\usepackage{stmaryrd}
\hypersetup{
    colorlinks=true,
    linkcolor=blue,
    filecolor=magenta,      
    urlcolor=blue,
}

\def\Name{Jérémy Meynier}  % Your name
\def\Login{} % nom du chapitre
\def\Homework{N} % Number of Homework
\def\Session{}




\author{\Name \texttt{\Login}}
\markboth{\Session\   \Name}{\Session\  \Name \texttt{\Login}}
\pagestyle{myheadings}
\date{}

\newenvironment{qparts}{\begin{enumerate}[{(}a{)}]}{\end{enumerate}}
\def\endproofmark{$\Box$}
\newenvironment{proof}{\par{\bf Proof}:}{\endproofmark\smallskip}

\textheight=9in
\textwidth=6.5in
\topmargin=-.75in
\oddsidemargin=0.25in
\evensidemargin=0.25in





% -----------------------------------------------------------------
\title{Espaces vectoriels normés}
\begin{document}
\maketitle

\section*{Exercice 1}

Soit $(a,b)\in\mathbb{R}^2$ et $f,g$ deux fonctions continues sur $[a,b]$. On suppose que $\forall x\in[a,b], f(x)>g(x)$.

Montrer qu'il existe $\lambda\in\mathbb{R^{+*}}$ tel que $\forall x\in[a,b], f(x)>g(x)+\lambda$

\section*{Exercice 2}

Soit $E$ un espace vectoriel normé, et $a_1,\dots,a_n$ des éléments de $E$.

Montrer que $\{x\in E, \prod_{k=1}^n ||x-a_k||=1\}$ est fermé borné dans $E$

\section*{Exercice 3}

On considère dans $\mathbb{R}^3$ la suite $(Z_n)_{n\in\mathbb{N}}$ définie par $Z_0=(U_0,V_0,W_0)\in\mathbb{R}^3$ et $\forall n\in\mathbb{N}$,

$ \begin{cases}
U_{n+1}=\frac{1}{3}U_n+\frac{1}{6}W_n+\frac{1}{2} \vspace{3mm} \\
V_{n+1}=\frac{1}{3}U_n+\frac{1}{6}V_n+\frac{1}{3}W_n \vspace{3mm}\\
W_{n+1}=\frac{1}{3}U_n+\frac{1}{3}V_n+\frac{1}{6}W_n-\frac{7}{6}
\end{cases}$ et $Z_n=(U_n,V_n,W_n)$

\begin{enumerate}
\item Montrer que $(Z_n)$ vérifie une relation matricielle de la forme $Z_{n+1} = AZ_n+B$
\item Montrer que $\exists k\in ]0,1[$/ $\forall X\in \mathbb{R}^3$, $||AX||_{\infty}\leq k||X||_{\infty}$
\item Montrer que $X=AX+B$ a une solution $L$ dans $\mathbb{R}^3$, puis montrer que $(Z_n)$ converge après avoir majoré $||Z_n-L||_{\infty}$ à l'aide de $||Z_0-L||_{\infty}$, de $k$ et $n$
\end{enumerate}

\section*{Exercice 4}

Soit $A$ un convexe d'un $\mathbb{K}$-espace vectoriel $E$ et $n\in \mathbb{N}$/ $n\geq 2$. Montrer que si $x_1,\cdots,x_n$ sont dans $A$ et si $\lambda_1,\cdots,\lambda_n$ sont des réels positifs tels que $\sum_{k=1}^n \lambda_k =1$, alors $\sum_{k=1}^n \lambda_k x_k \in A$

\section*{Exercice 5}

Soient $A,B\in M_p(\mathbb{K})$ et $(A_n)$ une suite de $GL_p(\mathbb{K})$ tel que $\lim_{n\to \infty} A_n = A$ et $\lim_{n\to \infty}A_n^{-1} = B$. Montrer que $A$ est inversible et que $A^{-1} = B$. On admettra que pour la norme $||A||=\sqrt{\operatorname{Tr}(A^TA)}$, $||AB||\leq ||A||\hspace{0.5mm}||B||$

\section*{Exercice 6}

Montrer que $GL_n(\mathbb{R})$ est dense dans $M_n(\mathbb{R})$.


\end{document}