\documentclass[11pt]{article}
\everymath{\displaystyle}
\usepackage[utf8]{inputenc}  
\usepackage[T1]{fontenc}
\usepackage[francais]{babel}
\usepackage{amsmath,textcomp,amssymb,geometry,graphicx,enumerate}
\usepackage{algorithm} % Boxes/formatting around algorithms
\usepackage[noend]{algpseudocode} % Algorithms
\usepackage{hyperref}
\usepackage{stmaryrd}
\hypersetup{
    colorlinks=true,
    linkcolor=blue,
    filecolor=magenta,      
    urlcolor=blue,
}

\def\Name{Jérémy Meynier}  % Your name
\def\Login{} % nom du chapitre
\def\Homework{N} % Number of Homework
\def\Session{}




\author{\Name \texttt{\Login}}
\markboth{\Session\   \Name}{\Session\  \Name \texttt{\Login}}
\pagestyle{myheadings}
\date{}

\newenvironment{qparts}{\begin{enumerate}[{(}a{)}]}{\end{enumerate}}
\def\endproofmark{$\Box$}
\newenvironment{proof}{\par{\bf Proof}:}{\endproofmark\smallskip}

\textheight=9in
\textwidth=6.5in
\topmargin=-.75in
\oddsidemargin=0.25in
\evensidemargin=0.25in



% -----------------------------------------------------------------
\title{Séries entières}
\begin{document}
\maketitle


\section*{Exercice 1}

Trouver le rayon de convergence $R$ de $\sum_{n\geq1} \frac{\sinh(n)}{n(n+1)}x^n$, puis calculer la somme pour $x\in ]-R,R[$

\section*{Exercice 2}

Soit $f(x) = \sum_{n=0}^\infty n^{(-1)^n}x^n$. Calculer le rayon de convergence et la somme sur le bon intervalle.

\section*{Exercice 3}

Soit $S(x) = \sum_{p=0}^\infty \frac{x^{3p}}{(3p)!} $. Donner l'ensemble de définition de $S$ et l'exprimer à l'aide de fonctions usuelles.

\section*{Exercice 4}

Déterminer le rayon de convergence des séries entières suivantes :

\begin{enumerate}
\item $\sum e^{-n}z^{2n}$
\item $\sum \sin(n)z^{n}$
\item $\sum z^{n^2}$
\item $\sum a_n z^{n}$ avec $a_{n+1}=a_n + \frac{2}{n+1}a_{n-1}$ et $ a_0=a_1=1$
\end{enumerate}

\section*{Exercice 5}

Déterminer le développement en séries entières en $0$ des fonctions suivantes:

\begin{enumerate}
\item $f(x) = \ln(x^2 +x +1)$
\item $f(x)= \ln(x^2 -5x +6)$
\item $f(x) = \frac{2x-1}{(2+x-x^2)^2}$
\end{enumerate}


\section*{Exercice 6}

Soit $f$ définie sur $]-1,1[$ par $f(x)=\frac{\arcsin(x)}{\sqrt{1-x^2}}$

\begin{enumerate}
\item Montrer que $f$ est développable en série entière sur $]-1,1[$
\item Montrer que $f$ est solution de $(1-x^2)y' - xy=1$
\item En déduire le développement en série entière de $f$ sur $]-1,1[$
\end{enumerate}


\end{document}