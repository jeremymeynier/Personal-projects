\documentclass[11pt]{article}
\everymath{\displaystyle}
\usepackage[utf8]{inputenc}  
\usepackage[T1]{fontenc}
\usepackage[francais]{babel}
\usepackage{amsmath,textcomp,amssymb,geometry,graphicx,enumerate}
\usepackage{algorithm} % Boxes/formatting around algorithms
\usepackage[noend]{algpseudocode} % Algorithms
\usepackage{hyperref}
\usepackage{stmaryrd}
\hypersetup{
    colorlinks=true,
    linkcolor=blue,
    filecolor=magenta,      
    urlcolor=blue,
}

\def\Name{Jérémy Meynier}  % Your name
\def\Login{} % nom du chapitre
\def\Homework{N} % Number of Homework
\def\Session{}




\author{\Name \texttt{\Login}}
\markboth{\Session\   \Name}{\Session\  \Name \texttt{\Login}}
\pagestyle{myheadings}
\date{}

\newenvironment{qparts}{\begin{enumerate}[{(}a{)}]}{\end{enumerate}}
\def\endproofmark{$\Box$}
\newenvironment{proof}{\par{\bf Proof}:}{\endproofmark\smallskip}

\textheight=9in
\textwidth=6.5in
\topmargin=-.75in
\oddsidemargin=0.25in
\evensidemargin=0.25in



% -----------------------------------------------------------------
\title{Réduction}
\begin{document}
\maketitle

\section*{Exercice 1}


Soit $A=(a_{ij})_{1\le{i,j}\le n} \in \textit{M}_n(\mathbb{R})$ avec $a_{ij}\in[0,1]$ et $\forall i \in \llbracket 1,n \rrbracket, \; \sum_{j=1}^n a_{ij} = 1$

\begin{enumerate}
\item Montrer que $1$ est valeur propre de $A$
\item Soit $\lambda$ une valeur propre de $A$. Montrer que $|\lambda|\le1$
\end{enumerate}

\section*{Exercice 2}

Soit $A =\begin{pmatrix}
1 & \dots & \dots & \dots & 1\\
\vdots & 0 & \dots & 0 & \vdots\\
\vdots & \vdots &\ddots  & \vdots & \vdots\\
\vdots & 0 & \dots & 0 & \vdots\\
1 & \dots & \dots & \dots & 1
\end{pmatrix} \in \emph{M}_n(\mathbb{R})$. Déterminer les éléments propres de $A$ et la diagonaliser.

\section*{Exercice 3}

Soit $A=\begin{pmatrix}
3 & -2 & 3\\
1 & 0 & 2\\
0 & 0 & 0
\end{pmatrix}$

\begin{enumerate}
\item Valeurs propres de $A$? Est-elle diagonalisable?
\item $A$ est-elle inversible?
\item Montrer que $A$ est semblable à $T=\begin{pmatrix}
1 & 0 & 0\\
0 & 2 & 1\\
0 & 0 & 2
\end{pmatrix}$
\item Calculer $T^n$ en fonction de $n$
\item En déduire $A^n$ en fonction de $n$
\end{enumerate}

\section*{Exercice 4}

Montrer que $M$ est nilpotente $\Leftrightarrow Sp_\mathbb{C}(M)=\{0\}$

\section*{Exercice 5}

Soit $A\in \emph{M}_n(\mathbb{R})$ tel que $A^3 + A =0$. Montrer que le rang de $A$ est pair. Même question si $A^3 +A^2 +A =0$

\section*{Exercice 6}

Déterminer le polynôme caractéristique d'un endomorphisme $f$ de rang $1$ d'un $\mathbb{K}$-ev $E$ de dimension $n$

\section*{Exercice 7}

Soit $u\in \emph{L}(E),\: E$ un $\mathbb{K}$-ev de dimension $n$. On suppose que $rg(u-id)=1$. Montrer que $u$ est diagonalisable $\Leftrightarrow \operatorname{Tr}(u)\ne n \Leftrightarrow \det(u) \ne 1$



\section*{Exercice 8}

Soit $A\in \emph{M}_n(\mathbb{R})$ tel que $A(A-I)^2 =0, \: A(A-I)\ne0, \:(A-I)^2\ne0.\: A$ est-elle diagonalisable?

\section*{Exercice 9}

Montrer que $M$ nilpotent $\Leftrightarrow \operatorname{Tr}(M^k)=0 \;\forall k \ge1$

\section*{Exercice 10}

Soit $A\in \emph{M}_n(\mathbb{R})$ tel que $A^2=-I_n$.\; Montrer que $\det(A)=1$

\section*{Exercice 11}

Déterminer les $A\in \emph{M}_n(\mathbb{R})$ tel que $\operatorname{Tr}(A)=n$ et $A^5=A^3$

\section*{Exercice 12}

Soient $A,B\in \emph{M}_n(\mathbb{C})$ tel que $AB=0$. Montrer que $A$ et $B$ admettent un vecteur propre commun 

\section*{Exercice 13}
Soit $A\in \emph{M}_n(\mathbb{R})$ tel que $A^3 -3A+4I_n =0$.
\begin{enumerate}
\item Montrer que $A\in \emph{GL}_n(\mathbb{R})$
\item Déterminer le signe de $\det(A)$
\end{enumerate}


\section*{Exercice 14}


Soit $E$ un espace vectoriel de dimension finie, et $f\in \textit{L}(E)$ de rang $1$.
\begin{enumerate}
\item Montrer que $f$ est diagonalisable si et seulement si $\operatorname{Tr}(f)\neq0$.
\item Montrer que $f$ est diagonalisable si et seulement si $f^2\neq0$
\end{enumerate}
  

\section*{Exercice 15}

Soit $M=\begin{pmatrix}
0 & 1 & \dots & 1\\
1 & \ddots & \ddots & \vdots\\
\vdots &  \ddots & \ddots & 1\\
1 & \dots & 1 & 0
\end{pmatrix}$

\begin{enumerate}
\item Déterminer les valeurs propres de $M$
\item Calculer $\det(M)$ et $M^{-1}$
\end{enumerate}

\section*{Exercice 16}

Soit $A=\begin{pmatrix}
1 & 3 & 0\\
3 & -2 & -1\\
0 & -1 & 1
\end{pmatrix}$.
Trouver les $M$ tels que $M^2=A$ dans $\textit{M}_n(\mathbb{R})$ puis dans $\textit{M}_n(\mathbb{C})$

\section*{Exercice 17}

Soit $A\in \textit{M}_n(\mathbb{R})$. Montrer que $\det(A^2 +I_n)\geq0$

\section*{Exercice 18}

Soit $E$ un espace vectoriel de dimension finie, et $f\in \textit{L}(E)$.
\begin{enumerate}
\item Montrer que $f$ admet un polynôme annulateur non nul
\item Montrer que $f$ est un automorphisme si et seulement si $f$ possède un polynôme annulateur $P$ tel que $P(0)\neq0$

\end{enumerate}



\end{document}