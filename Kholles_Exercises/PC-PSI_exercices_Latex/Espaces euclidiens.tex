\documentclass[11pt]{article}
\everymath{\displaystyle}
\usepackage[utf8]{inputenc}  
\usepackage[T1]{fontenc}
\usepackage[francais]{babel}
\usepackage{amsmath,textcomp,amssymb,geometry,graphicx,enumerate}
\usepackage{algorithm} % Boxes/formatting around algorithms
\usepackage[noend]{algpseudocode} % Algorithms
\usepackage{hyperref}
\usepackage{stmaryrd}
\hypersetup{
    colorlinks=true,
    linkcolor=blue,
    filecolor=magenta,      
    urlcolor=blue,
}

\def\Name{Jérémy Meynier}  % Your name
\def\Login{} % nom du chapitre
\def\Homework{N} % Number of Homework
\def\Session{}




\author{\Name \texttt{\Login}}
\markboth{\Session\   \Name}{\Session\  \Name \texttt{\Login}}
\pagestyle{myheadings}
\date{}

\newenvironment{qparts}{\begin{enumerate}[{(}a{)}]}{\end{enumerate}}
\def\endproofmark{$\Box$}
\newenvironment{proof}{\par{\bf Proof}:}{\endproofmark\smallskip}

\textheight=9in
\textwidth=6.5in
\topmargin=-.75in
\oddsidemargin=0.25in
\evensidemargin=0.25in



% -----------------------------------------------------------------
\title{Espaces euclidiens}
\begin{document}
\maketitle

\section*{Exercice 1}

Soit $E$ un espace euclidien et $ f : E \mapsto E$ une application vérifiant $f(0)=0$ et, $\forall (x,y)\in E^2,\: ||f(x) - f(y)|| = ||x-y||$ 

\begin{enumerate}
\item Montrer que $ \forall x\in E, ||f(x)||=||x||$
\item Montrer que $ \forall x\in E, f(-x)=-f(x)$
\item Montrer que $ \forall (x,y)\in E^2
, <f(x),f(y)>\, =\, <x,y>$
\item Soit $ \beta=(e_1,\dots,e_n)$ une base orthonormée de $E$. Montrer que $\forall x\in E,\\ f(x) = \sum_{k=1}^n <e_k,x>f(e_k)$
\item En déduire que $f$ est un automorphisme orthogonal de $E$

\end{enumerate}


\section*{Exercice 2}

Dans $\mathbb{R}^4$ muni de sa structure euclidienne canonique, on donne F :
$ \left\{
\begin{array}{l}
  x + y + z + t =0 \\
  x + 2y + 3z + 4t =0 \\
\end{array}
\right.$
\begin{enumerate}
\item Déterminer une baser orthonormée de $F$.
\item Donner la matrice dans $\mathbb{R}^4$ de la projection orthogonale de $p_F$ sur $F$.
\item Calculer $d(u,F)$ où $u=(1,1,1,1)$
\end{enumerate}

\section*{Exercice 3}

Soit $E$ un espace euclien, $E_1$ et $E_2$ deux sous-espaces de $E$. Montrer que $ (E_1 + E_2)^{\perp}= E_1^{\perp}\cap E_2^{\perp}$ et $(E_1\cap E_2)^{\perp} = E_1^{\perp} + E_2^{\perp}$

\section*{Exercice 4}

Déterminer $I = \inf_{ (a,b)\in\mathbb{R}^2}\left\{ \int_0^1 (t^3 -at-b)^2 \mathrm{d}t \right\}$

\section*{Exercice 5}

Montrer que $<A,B> \; = \operatorname{Tr}(A^{T}B)$ est un produit scalaire sur $\textit{M}_n(\mathbb{R})$, puis déterminer $\\\inf_{(a,b)\in\mathbb{R}^2} \left\{||M-aI-bJ||^2\right\}$ avec $J = \begin{pmatrix}
1 & \dots & 1\\
\vdots & \ddots & \vdots\\
1 & \dots & 1 
\end{pmatrix}$

\section*{Exercice 6}

Soient $(E,<,>)$ un espace euclidien et $a\in E$ tel que $||a||=1$. Si $\alpha\in\mathbb{R}$, on pose $\\ f_{\alpha}(x) = x + \alpha<a,x>a$

\begin{enumerate}

\item Soient $(\alpha,\beta)\in\mathbb{R}^2$. Calculer $f_\alpha\circ f_\beta$. Pour quels $\alpha$ $f_\alpha$ est-il bijectif?
\item Soit $\alpha\in\mathbb{R}$. Déterminer les éléments propres de $f_{\alpha}$
\item Pour quels $\alpha$ $f_\alpha$ est-il un automorphisme orthogonal de $E$?
\item Pour quels $\alpha$ $f_\alpha$ est-il un endomorphisme symétrique de $E$?
\end{enumerate}

\section*{Exercice 7}

Soit $u$ un endomorphisme symétrique d'un espace euclidien $E$ de valeurs propres $\lambda_1,\dots,\lambda_n$ comptées avec multiplicité et rangées en ordre croissant.

Montrer que $ \forall x\in E$, $\lambda_1||x||^2\leqslant \; <u(x),x>\;\leqslant  \lambda_n ||x||^2$  

\section*{Exercice 8}

Soit $A\in \textit{M}_n(\mathbb{R}
)$. Montrer que la matrice $A^{T}A$ est diagonalisable à valeurs propres positives

\end{document}