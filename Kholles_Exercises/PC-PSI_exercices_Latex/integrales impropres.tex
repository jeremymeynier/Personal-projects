\documentclass[11pt]{article}
\everymath{\displaystyle}
\usepackage[utf8]{inputenc}  
\usepackage[T1]{fontenc}
\usepackage[francais]{babel}
\usepackage{amsmath,textcomp,amssymb,geometry,graphicx,enumerate}
\usepackage{algorithm} % Boxes/formatting around algorithms
\usepackage[noend]{algpseudocode} % Algorithms
\usepackage{hyperref}
\usepackage{stmaryrd}
\hypersetup{
    colorlinks=true,
    linkcolor=blue,
    filecolor=magenta,      
    urlcolor=blue,
}

\def\Name{Jérémy Meynier}  % Your name
\def\Login{} % nom du chapitre
\def\Homework{N} % Number of Homework
\def\Session{}




\author{\Name \texttt{\Login}}
\markboth{\Session\   \Name}{\Session\  \Name \texttt{\Login}}
\pagestyle{myheadings}
\date{}

\newenvironment{qparts}{\begin{enumerate}[{(}a{)}]}{\end{enumerate}}
\def\endproofmark{$\Box$}
\newenvironment{proof}{\par{\bf Proof}:}{\endproofmark\smallskip}

\textheight=9in
\textwidth=6.5in
\topmargin=-.75in
\oddsidemargin=0.25in
\evensidemargin=0.25in



% -----------------------------------------------------------------
\title{Intégrales impropres/ Intégrales à paramètres}
\begin{document}
\maketitle

\section*{Exercice 1}

Soit $I_n= \int_0 ^{+\infty} \frac{\mathrm{d}x}{(1+x^4)^n}$. Prouver l'existence et donner la limite de $I_n$ 

\section*{Exercice 2}

Soit $I=\int_0^{+\infty} \frac{\sin(x)}{x}\mathrm{d}x$
\begin{enumerate}
\item $I$ est-elle convergente?
\item $I$ est-elle absolument convergente?
\end{enumerate}

\section*{Exercice 3}

Montrer que $\int_0^1 \frac{\ln(t)}{t-1}\mathrm{d}t = \sum_{k=1}^\infty \frac{1}{k^2}$

\section*{Exercice 4}

Soit $f: t\in ]0,1[ \mapsto \frac{1}{t^2} - \frac{1}{\arctan^2(t)}$

\begin{enumerate}
\item Montrer que $f$ est intégrable sur $]0,1]$
\item Donner un équivalent de $\int_x^1 \frac{1}{\arctan^2(t)}\mathrm{d}t$ quand $x\to 0^{+}$
\end{enumerate}

\section*{Exercice 5}

Soit $F(x) =\int_0^\infty \frac{\arctan(xt)}{t(1+t^2)}\mathrm{d}t$

\begin{enumerate}
\item Donner le domaine de définition de $F$
\item Étudier la continuité de $F$
\item Étudier le caractère $C^1$ de $F$
\item Trouver $F$ à l'aide d'une décomposition en éléments simples
\end{enumerate} 

\section*{Exercice 6}

On pose $f(x)=\int_x^\infty \frac{e^{-t}}{t}\mathrm{d}t$

\begin{enumerate}
\item Chercher $\lim_{x\to {+\infty}}$
\item À l'aide d'une intégration par parties, donner un équivalent de $f$ en $+\infty$
\item Déterminer un équivalent de $f(x)$ pour $x\to 0^{+}$
\end{enumerate}

\section*{Exercice 7}

Soient $a,b \in \mathbb{R^{+*}}$. Montrer que $\int_0^1 \frac{t^{a-1}}{1+t^b}\mathrm{d}t = \sum_{n=0}^{+\infty} \frac{(-1)^n}{a+nb}$

\section*{Exercice 8}

Étudier l'intégrabilité de $f$ sur $I$ dans les cas suivants :

\begin{enumerate}
\item $ f(t) =\frac{1}{e^t +t^2e^{-t}}, \;I=\mathbb{R^{+*}}$
\item $ f(t)=\frac{t^\alpha -1}{\ln(t)},\; I=]0,1[$
\item $ f(t)=\frac{1}{\arccos(t)},\; I=[0,1[$
\item $ f(t)=\frac{\sqrt{t}\sin(\frac{1}{t^2})}{\ln(1+t)},\; I=\mathbb{R^{+*}}$
\end{enumerate}

\section*{Exercice 9}

Pour $x>0$ et $n\in\mathbb{N^*}$, on pose $I_n(x)=\displaystyle\int_0^{+\infty} \frac{\mathrm{d}t}{(x+t^2)^n}$

\begin{enumerate}
\item Calculer $I_1(x)$
\item Montrer que $I_n\in C^1(]0,+\infty[)$
\item Calculer $I'_n (x)$ en fonction de $I_{n+1}(x)$
\item En déduire $I_n(x)$
\end{enumerate}

\end{document}