\documentclass[11pt]{article}
\everymath{\displaystyle}
\usepackage[utf8]{inputenc}
\usepackage{dsfont} 
\usepackage[T1]{fontenc}
\usepackage[francais]{babel}
\usepackage{amsmath,textcomp,amssymb,geometry,graphicx,enumerate}
\usepackage{algorithm} % Boxes/formatting around algorithms
\usepackage[noend]{algpseudocode} % Algorithms
\usepackage{hyperref}
\usepackage{stmaryrd}
\hypersetup{
    colorlinks=true,
    linkcolor=blue,
    filecolor=magenta,      
    urlcolor=blue,
}

\def\Name{Jérémy Meynier}  % Your name
\def\Login{} % nom du chapitre
\def\Homework{N} % Number of Homework
\def\Session{}




\author{\Name \texttt{\Login}}
\markboth{\Session\   \Name}{\Session\  \Name \texttt{\Login}}
\pagestyle{myheadings}
\date{}

\newenvironment{qparts}{\begin{enumerate}[{(}a{)}]}{\end{enumerate}}
\def\endproofmark{$\Box$}
\newenvironment{proof}{\par{\bf Proof}:}{\endproofmark\smallskip}

\textheight=9in
\textwidth=6.5in
\topmargin=-.75in
\oddsidemargin=0.25in
\evensidemargin=0.25in



% -----------------------------------------------------------------
\title{Probabilités}
\begin{document}
\maketitle

\section*{Exercice 1}

Soient $X$ et $Y$ deux variables aléatoires réelles indépendantes, de lois binomiales $B(n,p)$ et $B(m,p)$ respectivement. On pose $S=X+Y$. Calculer $\mathrm{P}(X=x|S=s)$

\section*{Exercice 2}

Lors d'une rencontre d'athlétisme, la barre est montée d'un cran après chaque saut réussi par le concurrent. Quand un saut est raté, la compétition d'arrête pour le sauteur. Celui-là a, pour le saut $n$, une chance sur $n$ de réussir le saut. On appelle $X$ le rang du dernier saut réussi. 

\begin{enumerate}
\item Quelle est la loi de $X$?
\item Vérifier qu'il s'agit d'une variable aléatoire
\item Calculer l'espérance et la variance de $X$ 
\end{enumerate}

\section*{Exercice 3}

Soit $\lambda>0, a>0$ et $a\neq1$. On suppose que $(X,Y)$ a une loi de probabilité définie par $\mathrm{P}(X=i\cap Y=j)=\frac{\lambda a^i}{j!},\; 0\leq i\leq j$

\begin{enumerate}
\item Déterminer la loi de $Y$
\item Déterminer $\lambda$ pour qu'il s'agisse bien d'une loi de couples.
\item $X$ et $Y$ sont-elles indépendantes?
\end{enumerate}
~
\section*{Exercice 4}

Soient $X$ et $Y$ deux variables aléatoires réelles indépendantes. $Y\sim P(\lambda), X$ suit une loi uniforme sur $\{1,2\}$, et $Z=XY$.

\begin{enumerate}
\item Déterminer la loi de $Z$
\item Calculer $\mathbb{E}(Z)$ et $\operatorname{Var}(Z)$
\item Calculer la probabilité que $Z$ soit paire
\end{enumerate}

\section*{Exercice 5}

Soient $X$ et $Y$ deux variables aléatoires réelles indépendantes suivant une loi géométrique de paramètres respectifs $a$ et $b$, et $Z=Y-X$. 

\begin{enumerate}
\item Trouver la loi de $Z$
\item Calculer $\mathrm{P}(X\leq Y)$
\end{enumerate}

\section*{Exercice 6}

On lance une pièce équilibrée consécutivement. On s'arrête dès que deux piles successifs sont apparus. $X$ est le nombre de lancers jusqu'à l'arrêt. Trouver la loi de $X$

\section*{Exercice 7}

Soient $X$, $Y$ des variables aléatoires indépendantes suivant une loi de Poisson de paramètre $\lambda \in \mathbb{R}^{+*}$.

\begin{enumerate}
\item Déterminer la fonction génératrice de $X$ et $3Y$
\item En déduire la fonction génératrice de $Z = X + 3Y$
\item En déduire $\mathbb{E}(Z)$ et $\mathbb{V}(Z)$
\item $X$ et $Z$ sont-elles indépendantes? Donner alors le coefficient de correlation linéraire $\rho(X,Z)$
\item Montrer que la loi de $Z$ s'écrit $\mathbb{P}(Z = n) = A_ne^{-2\lambda}$ avec $A_n$ ne faisant pas intervenir d'exponentielle.
\item Trouver le minimum de $f(t) = \mathbb{V}(X + tZ)$
\end{enumerate}

\section*{Exercice 8}

Un point $M$ se déplace dans un plan muni d'un repère$(O,\vec{i},\vec{j}$. Au départ, $M$ est au point $O$. À chaque instant il se déplace d'un pas dans l'une des quatres directions $(\vec{i},\vec{-i},\vec{j},\vec{-j})$. Ses coordonnées après $n$ déplacements sont des variables aléatoires réelles $X_n,Y_n$.

\begin{enumerate}
\item Calculer $\mathbb{P}(X_n=n),\mathbb{P}(Y_n=n),\mathbb{P}(Y_n=n\cap X_n=n)$. $X_n$ et $Y_n$ sont-elles indépendantes?
\item Trouver une relation entre $\mathbb{E}(X_{n+1}^2)$ et $\mathbb{E}(X_n^2)$. Calculer $\mathbb{E}(X_n^2)$.
\item Calculer $\mathbb{P}(Y_n=0\cap X_n=0)$
\end{enumerate}


\end{document}