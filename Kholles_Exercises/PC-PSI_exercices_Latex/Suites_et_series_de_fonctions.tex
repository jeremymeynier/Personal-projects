\documentclass[11pt]{article}
\everymath{\displaystyle}
\usepackage[utf8]{inputenc}  
\usepackage[T1]{fontenc}
\usepackage[francais]{babel}
\usepackage{amsmath,textcomp,amssymb,geometry,graphicx,enumerate}
\usepackage{algorithm} % Boxes/formatting around algorithms
\usepackage[noend]{algpseudocode} % Algorithms
\usepackage{hyperref}
\usepackage{stmaryrd}
\hypersetup{
    colorlinks=true,
    linkcolor=blue,
    filecolor=magenta,      
    urlcolor=blue,
}

\def\Name{Jérémy Meynier}  % Your name
\def\Login{} % nom du chapitre
\def\Homework{N} % Number of Homework
\def\Session{}




\author{\Name \texttt{\Login}}
\markboth{\Session\   \Name}{\Session\  \Name \texttt{\Login}}
\pagestyle{myheadings}
\date{}

\newenvironment{qparts}{\begin{enumerate}[{(}a{)}]}{\end{enumerate}}
\def\endproofmark{$\Box$}
\newenvironment{proof}{\par{\bf Proof}:}{\endproofmark\smallskip}

\textheight=9in
\textwidth=6.5in
\topmargin=-.75in
\oddsidemargin=0.25in
\evensidemargin=0.25in



% -----------------------------------------------------------------
\title{Suites et series de fonctions}
\begin{document}
\maketitle

\section*{Exercice 1}

Étudier les convergences (simples, uniformes, normales) des suites et séries de fonctions $(f_n)_{n\in\mathbb{N}}$ suivantes:

\begin{enumerate}
\item $ f_n(x)=\frac{nx^2}{1+nx}$, $I=[0,1]$
\item $ f_n(x)=x^2 e^{-nx}$, $I=\mathbb{R^{+}}$
\item $ f_n(x)= n\ln(1+\frac{1}{nx})$, $I=]0,+\infty[$

\end{enumerate}

\section*{Exercice 2}

Étudier les convergences (simples, uniformes, normales) des séries de fonctions suivantes

\begin{enumerate}
\item $\sum_{n\geq1} x^n\ln^2(x)$
\item $\sum_{n\geq1} x^n\ln(x)$
\end{enumerate}

\section*{Exercice 3}

Soit $f_n(x)=\frac{\sin(nx)}{1+n^2x^2}$

\begin{enumerate}
\item Étudier les convergences de $f_n$
\item Montrer que $\forall a>0 \;\sum f_n$ converge normalement sur $[a, +\infty[$
\item Montrer que $ \sum f_n$ ne converge pas normalement sur $]0, +\infty[$
\end{enumerate}

\section*{Exercice 4}

Soit $f_n(x)=\frac{x}{(1+x^2)^n}$. Étudier les convergences de $(f_n)_{n\geq1}$, puis de $\sum_{n\geq1} f_n$

\section*{Exercice 5}

Soit $(f_n)_{n\in\mathbb{N^*}}$ définie sur $[0,2]$ par $f_n(x)=n^2x(1-x)^n$.

\begin{enumerate}
\item Étudier les variations de $f_n$ sur $[0,2]$
\item Calculer $\lim_{n\to +\infty} \int_0^1 f_n(x)\mathrm{d}x$
\item Étudier les convergences de la suite $(f_n)_{n\in\mathbb{N^*}}$ sur $[0,1]$
\item Montrer que $\forall a\in ]0,1[$,
 la suite $(f_n)_{n\in\mathbb{N^*}}$ converge uniformément sur $[a,2-a]$
\end{enumerate}

\section*{Exercice 6}

Soit $\phi(x) = \sum_{n=1}^\infty \frac{(-1)^{n+1}}{n^x}$
\begin{enumerate}
\item Montrer que $\phi$ converge uniformément sur tout segment de $]0,+\infty[$
\item Montrer que $\phi$ est continue sur $]0,+\infty[$
\item Montrer que $\phi$ ne converge pas uniformément sur $]0,+\infty[$
\end{enumerate}

\section*{Exercice 7}

soit $\zeta(x)=\sum_{n=1}^{+\infty} \frac{1}{n^x}$
\begin{enumerate}
\item Donner le domaine de définition de $\zeta$
\item Étudier la convergence normale et la converge uniforme de $\zeta$
\item Montrer que $\zeta\in C^0(]1,+\infty[)$
\item Déterminer un équivalent de $\zeta$ en $1^{+}$
\item Déterminer la limite de $\zeta$ en $+\infty$
\end{enumerate}

\end{document}