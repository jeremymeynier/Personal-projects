\documentclass[11pt]{article}
\everymath{\displaystyle}
\usepackage[utf8]{inputenc}  
\usepackage[T1]{fontenc}
\usepackage[francais]{babel}
\usepackage{amsmath,textcomp,amssymb,geometry,graphicx,enumerate}
\usepackage{algorithm} % Boxes/formatting around algorithms
\usepackage[noend]{algpseudocode} % Algorithms
\usepackage{hyperref}
\usepackage{stmaryrd}
\hypersetup{
    colorlinks=true,
    linkcolor=blue,
    filecolor=magenta,      
    urlcolor=blue,
}

\def\Name{Jérémy Meynier}  % Your name
\def\Login{} % nom du chapitre
\def\Homework{N} % Number of Homework
\def\Session{}




\author{\Name \texttt{\Login}}
\markboth{\Session\   \Name}{\Session\  \Name \texttt{\Login}}
\pagestyle{myheadings}
\date{}

\newenvironment{qparts}{\begin{enumerate}[{(}a{)}]}{\end{enumerate}}
\def\endproofmark{$\Box$}
\newenvironment{proof}{\par{\bf Proof}:}{\endproofmark\smallskip}

\textheight=9in
\textwidth=6.5in
\topmargin=-.75in
\oddsidemargin=0.25in
\evensidemargin=0.25in




% -----------------------------------------------------------------
\title{Calcul différentiel}
\begin{document}
\maketitle

\section*{Exercice 1}

Résoudre $ x\frac{\partial f}{\partial x}+y\frac{\partial f}{\partial y} = \frac{y}{x}$

\section*{Exercice 2}

Résoudre $ \frac{\partial ^2 f}{\partial x^2 } - \frac{\partial ^2 f}{\partial y^2 } = 0$ à l'aide du changement de variable $u=x-y,\;  v=x+y$

\section*{Exercice 3}

Résoudre $ x^2\frac{\partial ^2 f}{\partial x^2} + 2xy\frac{\partial ^2 f}{\partial y\partial x} + y^2\frac{\partial ^2 f}{\partial y^2} = 0$ en posant $u=x, \; v=\frac{y}{x}$

\section*{Exercice 4}

Résoudre $ (x+y)\frac{\partial f}{\partial x} + (x-y)\frac{\partial f}{\partial y} = 0$ en posant $u = x^2-y^2-2xy, \; v=y$

\end{document}
