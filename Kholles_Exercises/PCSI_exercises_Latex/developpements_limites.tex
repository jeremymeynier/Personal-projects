\documentclass[11pt]{article}
\everymath{\displaystyle}
\usepackage[utf8]{inputenc}  
\usepackage[T1]{fontenc}
\usepackage[francais]{babel}
\usepackage{amsmath,textcomp,amssymb,geometry,graphicx,enumerate}
\usepackage{algorithm} % Boxes/formatting around algorithms
\usepackage[noend]{algpseudocode} % Algorithms
\usepackage{hyperref}
\usepackage{stmaryrd}
\hypersetup{
    colorlinks=true,
    linkcolor=blue,
    filecolor=magenta,      
    urlcolor=blue,
}

\def\Name{Jérémy Meynier}  % Your name
\def\Login{} % nom du chapitre
\def\Homework{N} % Number of Homework
\def\Session{}




\author{\Name \texttt{\Login}}
\markboth{\Session\   \Name}{\Session\  \Name \texttt{\Login}}
\pagestyle{myheadings}
\date{}

\newenvironment{qparts}{\begin{enumerate}[{(}a{)}]}{\end{enumerate}}
\def\endproofmark{$\Box$}
\newenvironment{proof}{\par{\bf Proof}:}{\endproofmark\smallskip}

\textheight=9in
\textwidth=6.5in
\topmargin=-.75in
\oddsidemargin=0.25in
\evensidemargin=0.25in





% -----------------------------------------------------------------
\title{Développements limités}
\begin{document}
\maketitle

\section*{Exercice 1}

Donner le développement limité à l'ordre $3$ au voisinage de $3$ de $g(x)=\sqrt{1+x}$

\section*{Exercice 2}

Donner le développement limité à l'ordre $5$ en $0$ de $f(x)=\exp(\tan(x))$

\section*{Exercice 3}

Soit $f(x)=\frac{1}{x^2}(\frac{\ln(1+x)+\ln(1-x)}{x^2}+1)$. Déterminer $\lim_{x\to0} f(x)$

\section*{Exercice 4}

Déterminer le développement asymptotique à l'ordre $4$ de $\ln(x+\sqrt{1+x^2})$

\section*{Exercice 5}
\begin{enumerate}
\item Montrer qu'il existe $x_n\in ]n\pi -\frac{\pi}{2}, n\pi +\frac{\pi}{2}[$ tel que $\tan(x_n)=x_n$
\item Donner un développement asymptotique à $3$ termes de $x_n$
\end{enumerate}

\section*{Exercice 6}

Soit $f:x\in\mathbb{R}^{+*} \mapsto x-\ln(x)$

\begin{enumerate}
\item Soit $n\geq2$. Montrer $f(x)=n$ possède $2$ solutions $u_n$ et $v_n$ avec $u_n\in]0,1[$ et $v_n\in ]1,+\infty[$
\item Montrer que $v_n= n+\ln(n)+\frac{\ln(n)}{n} + o(\frac{\ln(n)}{n})$
\item Montrer que $u_n= e^{-n} + e^{-2n} +o(e^{-2n})$
\end{enumerate}

\section*{Exercice 7}

Soit $f(x)=e^x +x$
\begin{enumerate}
\item Montrer que $\forall n\in\mathbb{N}^{*} \; f(x)=n$ possède une unique solution sur $\mathbb{R}^{+}$, que l'on nomme $u_n$
\item Donner un développement asymptotique à $2$ termes de $u_n$
\end{enumerate}

\section*{Exercice 8}

Soit $u_n=\sqrt{n+\sqrt{n-1+\sqrt{\dots\sqrt{1+\sqrt{0}}}}}$
\begin{enumerate}
\item Expliciter une relation entre $u_n$ et $u_{n-1}$
\item Faire un développement limité à l'ordre $3$ de $u_n$
\end{enumerate}

\end{document}