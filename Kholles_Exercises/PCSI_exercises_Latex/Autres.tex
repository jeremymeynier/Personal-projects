\documentclass[11pt]{article}
\everymath{\displaystyle}
\usepackage[utf8]{inputenc}  
\usepackage[T1]{fontenc}
\usepackage[francais]{babel}
\usepackage{amsmath,textcomp,amssymb,geometry,graphicx,enumerate}
\usepackage{algorithm} % Boxes/formatting around algorithms
\usepackage[noend]{algpseudocode} % Algorithms
\usepackage{hyperref}
\usepackage{stmaryrd}
\hypersetup{
    colorlinks=true,
    linkcolor=blue,
    filecolor=magenta,      
    urlcolor=blue,
}

\def\Name{Jérémy Meynier}  % Your name
\def\Login{} % nom du chapitre
\def\Homework{N} % Number of Homework
\def\Session{}




\author{\Name \texttt{\Login}}
\markboth{\Session\   \Name}{\Session\  \Name \texttt{\Login}}
\pagestyle{myheadings}
\date{}

\newenvironment{qparts}{\begin{enumerate}[{(}a{)}]}{\end{enumerate}}
\def\endproofmark{$\Box$}
\newenvironment{proof}{\par{\bf Proof}:}{\endproofmark\smallskip}

\textheight=9in
\textwidth=6.5in
\topmargin=-.75in
\oddsidemargin=0.25in
\evensidemargin=0.25in





% -----------------------------------------------------------------
\title{Autres (Applications/Dénombrement/Arithmétique)}
\begin{document}
\maketitle
\part*{Applications}

\section*{Exercice 1}

Soient $E,F,G$ trois ensembles, $f:E\mapsto F$ et $g:F\mapsto G$

\begin{enumerate}
\item Montrer que si $g\circ f$ est surjective alors $g$ est surjective
\item Montrer que si $g\circ f$ est injective alors $f$ est injective
\end{enumerate}

\section*{Exercice 2}

Soit $E$ un ensemble et $p:E\mapsto E$/ $p\circ p=p$
\begin{enumerate}
\item On suppose $p$ injective. Montrer que $p = id_E$
\item On suppose $p$ surjective. Montrer que $p = id_E$
\end{enumerate}

\section*{Exercice 3}

Soient $E$ et $F$ des ensembles, $f:E\mapsto F$ et $g:F\mapsto E$. On suppose que $g\circ f$ est bijective.
Montrer que $f$ est injective et que $g$ est surjective.

\section*{Exercice 4}

Soient $E,F,G$ des ensembles, $f:E\mapsto F$ et $g:F\mapsto G$.
\begin{enumerate}
\item On suppose $g\circ f$ injective et $f$ surjective. Montrer que $g$ est injective.
\item On suppose $g\circ f$ surjective et $g$ injective. Montrer que $f$ est surjective.

\end{enumerate}

\section*{Exercice 5}

Soit $E$ un ensemble et $f:E\mapsto E$/ $f\circ f\circ f=f$. Montrer que $f$ est injective si et seulement si $f$ est surjective.

\part*{Arithmétique}

\section*{Exercice 1}

Soit $p\geq 5$ un nombre premier. Montrer que $p^2 -1$ est divisible par 24.

\section*{Exercice 2}

Trouver le dernier chiffre de l'écriture décimale de $1997^{2001^{2003}}$

\section*{Exercice 3}

Trouver les deux derniers chiffres de l'écriture décimale de $2^{2018}$

\section*{Exercice 4}

Montrer, si $n\in\mathbb{N}$, que $4^n +15n -1\equiv 0[9]$

\section*{Exercice 5}

Montrer que, $\forall n\in\mathbb{N}$, $6|5n^3+n$

\section*{Exercice 6}

Montrer que, $\forall n\in\mathbb{N}$, $42|n^{13}-n$

\part*{Dénombrement}

\section*{Exercice 1}

Montrer avec un argument combinatoire que $\sum_{k=0}^r \dbinom{n}{k}\dbinom{m}{r-k}=\dbinom{n+m}{r}$

\section*{Exercice 2}

Montrer avec un argument combinatoire que $k\dbinom{n}{k}=n\dbinom{n-1}{k-1}$

\section*{Exercice 3}

Montrer avec un argument combinatoire que $\dbinom{n}{k}=\dbinom{n-1}{k-1}+\dbinom{n-1}{k}$

\section*{Exercice 4}



	

\end{document}