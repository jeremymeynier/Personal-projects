\documentclass[11pt]{article}
\everymath{\displaystyle}
\usepackage[utf8]{inputenc}  
\usepackage[T1]{fontenc}
\usepackage[francais]{babel}
\usepackage{amsmath,textcomp,amssymb,geometry,graphicx,enumerate}
\usepackage{algorithm} % Boxes/formatting around algorithms
\usepackage[noend]{algpseudocode} % Algorithms
\usepackage{hyperref}
\usepackage{stmaryrd}
\hypersetup{
    colorlinks=true,
    linkcolor=blue,
    filecolor=magenta,      
    urlcolor=blue,
}

\def\Name{Jérémy Meynier}  % Your name
\def\Login{} % nom du chapitre
\def\Homework{N} % Number of Homework
\def\Session{}




\author{\Name \texttt{\Login}}
\markboth{\Session\   \Name}{\Session\  \Name \texttt{\Login}}
\pagestyle{myheadings}
\date{}

\newenvironment{qparts}{\begin{enumerate}[{(}a{)}]}{\end{enumerate}}
\def\endproofmark{$\Box$}
\newenvironment{proof}{\par{\bf Proof}:}{\endproofmark\smallskip}

\textheight=9in
\textwidth=6.5in
\topmargin=-.75in
\oddsidemargin=0.25in
\evensidemargin=0.25in





% -----------------------------------------------------------------
\title{Déterminants}
\begin{document}
\maketitle

\section*{Exercice 1}
 
Calculer $\begin{vmatrix}
b & a & \dots & a\\
a & \ddots & \ddots & \vdots\\
\vdots & \ddots & \ddots & a\\
a & \dots & a & b
\end{vmatrix}$

\section*{Exercice 2}

$\forall n\in \mathbb{N}^*$, calculer le déterminant $D_n$ de la matrice de terme fénéral $|i-j|$, $1\leq i,j \leq n$ 

\section*{Exercice 3}

Calculer $D_n = \begin{vmatrix}
1 &  \cdots & 1\\
\vdots & \ddots & (0)\\
1 & (0) & 1\\
\end{vmatrix}$

\section*{Exercice 4}

Exprimer $D_n = \begin{vmatrix}
2 & 1 & \cdots & 1\\
1 & 3 & \ddots &  \vdots\\
\vdots & \ddots & \ddots & 1\\
1 & \cdots & 1 & n+1
\end{vmatrix}$ en fonction de $H_n=\sum_{k=1}^n \frac{1}{k}$.

\section*{Exercice 5}

\begin{enumerate}
\item Calculer $ \begin{vmatrix}
a & b & c \\
a^2 & b^2 & c^2 \\
a^3 & b^3 & c^3 \\
\end{vmatrix}$
\item En déduire $ \begin{vmatrix}
a+b & b+c & c+a \\
a^2+b^2 & b^2+c^2 & c^2+a^2 \\
a^3+b^3 & b^3+c^3 & c^3+a^3 \\
\end{vmatrix}$
\end{enumerate}

\section*{Exercice 6}

Soient $A,B\in M_n(\mathbb{R})$. Montrer que $U = \begin{vmatrix}
A & B\\
B & A\\
\end{vmatrix} = \det(A+B)\det(A-B)$.

\section*{Exercice 7}

Soit $H\in M_n(\mathbb{R})$ de rang 1. Montrer que $\det(A+H)\det(A-H)\leq \det(A^2)$





\end{document}