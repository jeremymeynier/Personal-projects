\documentclass[11pt]{article}
\everymath{\displaystyle}
\usepackage[utf8]{inputenc}  
\usepackage[T1]{fontenc}
\usepackage[francais]{babel}
\usepackage{amsmath,textcomp,amssymb,geometry,graphicx,enumerate}
\usepackage{algorithm} % Boxes/formatting around algorithms
\usepackage[noend]{algpseudocode} % Algorithms
\usepackage{hyperref}
\usepackage{stmaryrd}
\hypersetup{
    colorlinks=true,
    linkcolor=blue,
    filecolor=magenta,      
    urlcolor=blue,
}

\def\Name{Jérémy Meynier}  % Your name
\def\Login{} % nom du chapitre
\def\Homework{N} % Number of Homework
\def\Session{}




\author{\Name \texttt{\Login}}
\markboth{\Session\   \Name}{\Session\  \Name \texttt{\Login}}
\pagestyle{myheadings}
\date{}

\newenvironment{qparts}{\begin{enumerate}[{(}a{)}]}{\end{enumerate}}
\def\endproofmark{$\Box$}
\newenvironment{proof}{\par{\bf Proof}:}{\endproofmark\smallskip}

\textheight=9in
\textwidth=6.5in
\topmargin=-.75in
\oddsidemargin=0.25in
\evensidemargin=0.25in





% -----------------------------------------------------------------
\title{Polynômes}
\begin{document}
\maketitle

\section*{Exercice 1}

\begin{enumerate}
\item Montrer que $\exists!\; T_n\in\mathbb{R}[X] /\; T_n(\cos(\theta))=\cos(n\theta)$
\item Montrer que $\forall x\in [-1,1], T_{n+2}=2XT_{n+1}-T_n$
\item Donner le coefficient dominant et le degré de $T_n$
\item Donner une équation différentielle du second ordre vérifiée par $T_n$
\item Donner les racines de  $T_n$, puis le factoriser
\item Montrer que les $(T_n)_{n\in\mathbb{N}}$ forment une famille orthogonale
\end{enumerate}

\section*{Exercice 2}

Donner le reste de la division euclidienne de $(\cos(\theta)+X\sin(\theta))^n$ par $(X^2 +1)$

\section*{Exercice 3}

Trouver les polynômes $P\in\mathbb{R}[X]/\; P(X^2)=(X^2+1)P(X)$

\section*{Exercice 4}

Montrer que si $P\in\mathbb{R}[X]\backslash\{0\}$ vérifie $P(X^2)=P(X)P(X+1)$, ses racines sont parmi $\{0,1,-j,-j^2\}$. En déduire tous les polynômes solutions

\section*{Exercice 5}

Déterminer les polynômes $P\in\mathbb{R}[X]/\;(X+4)P(X)=XP(X+1)$

\section*{Exercice 6}

Déterminer les polynômes $P\in\mathbb{R}[X]/\;\forall n\in\mathbb{N} \int_n^{n+1} P(t)\mathrm{d}t = n^2 +1$

\end{document}