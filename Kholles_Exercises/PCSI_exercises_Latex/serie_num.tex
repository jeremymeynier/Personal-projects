\documentclass[11pt]{article}
\everymath{\displaystyle}
\usepackage[utf8]{inputenc}  
\usepackage[T1]{fontenc}
\usepackage[francais]{babel}
\usepackage{amsmath,textcomp,amssymb,geometry,graphicx,enumerate}
\usepackage{algorithm} % Boxes/formatting around algorithms
\usepackage[noend]{algpseudocode} % Algorithms
\usepackage{hyperref}
\usepackage{stmaryrd}
\hypersetup{
    colorlinks=true,
    linkcolor=blue,
    filecolor=magenta,      
    urlcolor=blue,
}

\def\Name{Jérémy Meynier}  % Your name
\def\Login{} % nom du chapitre
\def\Homework{N} % Number of Homework
\def\Session{}




\author{\Name \texttt{\Login}}
\markboth{\Session\   \Name}{\Session\  \Name \texttt{\Login}}
\pagestyle{myheadings}
\date{}

\newenvironment{qparts}{\begin{enumerate}[{(}a{)}]}{\end{enumerate}}
\def\endproofmark{$\Box$}
\newenvironment{proof}{\par{\bf Proof}:}{\endproofmark\smallskip}

\textheight=9in
\textwidth=6.5in
\topmargin=-.75in
\oddsidemargin=0.25in
\evensidemargin=0.25in




% -----------------------------------------------------------------
\title{Série numérique}
\begin{document}
\maketitle

\section*{Exercice 1}

Nature de la série de terme général $u_n=\sin(\frac{\pi n^2}{n+1})$ 
\section*{Exercice 2}

\begin{enumerate}
\item Pour quelles valeurs de $a,b\in \mathbb{R}$ la série de terme général $u_n=\ln(n) + a\ln(n+1) +b\ln(n+2)$ est-elle convergente?
\item Calculer la somme de cette série
\end{enumerate}


\section*{Exercice 3}

Nature des séries suivantes :

\begin{enumerate}

\item $u_n=\frac{1}{n^{\frac{3}{4}}\ln^2(n)}$
\item $u_n=(1+\frac{1}{\sqrt{n}})^{-n}$
\item $u_n=\ln(\frac{2+\sin(\frac{1}{n})}{2-\sin(\frac{1}{n})})$
\item $u_n=\frac{\ln^2(n)}{n^{\frac{5}{4}}}$

\end{enumerate}

\section*{Exercice 4}

Nature de la série de terme général $u_n=\ln(1+\frac{(-1)^n}{n^{\alpha}}),\; \alpha>0$

\section*{Exercice 5}

Soit $u_n>0$. On pose $S_n=\sum_{k=0}^{n} u_k$. On suppose que $\sum_{n=0}^\infty u_n$ converge.

Donner la nature des séries de termes généraux $\frac{u_n}{S_n}$ et $\frac{u_n}{S_n^2}$

\section*{Exercice 6}

\begin{enumerate}
\item Par comparaison à une intégrale, donner un équivalent de $u_n=\sum_{k=1}^{n} \ln^2(k)$
\item La série $ \sum_{n\ge2} \frac{1}{u_n}$ est-elle convergente? 
\end{enumerate}

\section*{Exercice 7}

Nature de la série de terme général $u_n=\frac{n^{\alpha}\ln(n)^n}{n!} , \alpha\in \mathbb{R}$

\section*{Exercice 8}

Caluler $\lim\limits_{n \rightarrow +\infty} \frac{1}{n+1}+\cdots+\frac{1}{2n}=\lim\limits_{n \rightarrow +\infty} \sum_{k=1}^{n} \frac{1}{n+k}$

\section*{Exercice 9}

\begin{enumerate}
\item Soit la série de terme général $u_n=\frac{1}{n\ln^{\alpha}(n)}, n\ge2$

Montrer que $\sum u_n$ diverge si $\alpha\ge1$ et converge si $\alpha>1$

\emph{Indication}: même démonstration que pour la série de Riemann

\item En déduire la nature de la série $\sum_{n=2}^{\infty} \frac{(e - (1+\frac{1}{n•})^n)e^\frac{1}{n}}{\ln^2(n^2+n)}$


\end{enumerate}

\end{document}