\documentclass[11pt]{article}
\everymath{\displaystyle}
\usepackage[utf8]{inputenc}  
\usepackage[T1]{fontenc}
\usepackage[francais]{babel}
\usepackage{amsmath,textcomp,amssymb,geometry,graphicx,enumerate}
\usepackage{algorithm} % Boxes/formatting around algorithms
\usepackage[noend]{algpseudocode} % Algorithms
\usepackage{hyperref}
\usepackage{stmaryrd}
\hypersetup{
    colorlinks=true,
    linkcolor=blue,
    filecolor=magenta,      
    urlcolor=blue,
}

\def\Name{Jérémy Meynier}  % Your name
\def\Login{} % nom du chapitre
\def\Homework{N} % Number of Homework
\def\Session{}




\author{\Name \texttt{\Login}}
\markboth{\Session\   \Name}{\Session\  \Name \texttt{\Login}}
\pagestyle{myheadings}
\date{}

\newenvironment{qparts}{\begin{enumerate}[{(}a{)}]}{\end{enumerate}}
\def\endproofmark{$\Box$}
\newenvironment{proof}{\par{\bf Proof}:}{\endproofmark\smallskip}

\textheight=9in
\textwidth=6.5in
\topmargin=-.75in
\oddsidemargin=0.25in
\evensidemargin=0.25in





% -----------------------------------------------------------------
\title{Int\'egration}
\begin{document}
\maketitle

\section*{Exercice 1}

Donner une primitive des fonctions suivantes:
\vspace{2ex}

\begin{enumerate}


\item $f(t)=t\arctan^2(t)$

\item $ f(t)=\frac{t+1}{(t-1)^2} $

\item $ f(t) = \frac{1}{\cos(t)+\sin(t)} $

\item $ f(t) = \frac{\cos(t)}{\cos(t)+\sin(t)} $
\end{enumerate}


\section*{Exercice 2}

Soit $f\in C^0([0,1],\mathbb{R})$. On définit $F:[0,1]\mapsto\mathbb{R}$ par $F(x)=\int_{0}^{1} \min(x,t)f(t)\, \mathrm{d}t$. Montrer que $F$ est $C^2$ et calculer $F^{(2)}(x)$

\section*{Exercice 3}

Déterminer la limite de la suite de terme général $u_n=\frac{1}{n^{p+1}}\sum_{k=1}^{n}k^p$


\end{document}
