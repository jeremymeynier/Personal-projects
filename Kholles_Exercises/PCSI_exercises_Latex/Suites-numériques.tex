\documentclass[11pt]{article}
\everymath{\displaystyle}
\usepackage[utf8]{inputenc}  
\usepackage[T1]{fontenc}
\usepackage[francais]{babel}
\usepackage{amsmath,textcomp,amssymb,geometry,graphicx,enumerate}
\usepackage{algorithm} % Boxes/formatting around algorithms
\usepackage[noend]{algpseudocode} % Algorithms
\usepackage{hyperref}
\usepackage{stmaryrd}
\hypersetup{
    colorlinks=true,
    linkcolor=blue,
    filecolor=magenta,      
    urlcolor=blue,
}

\def\Name{Jérémy Meynier}  % Your name
\def\Login{} % nom du chapitre
\def\Homework{N} % Number of Homework
\def\Session{}




\author{\Name \texttt{\Login}}
\markboth{\Session\   \Name}{\Session\  \Name \texttt{\Login}}
\pagestyle{myheadings}
\date{}

\newenvironment{qparts}{\begin{enumerate}[{(}a{)}]}{\end{enumerate}}
\def\endproofmark{$\Box$}
\newenvironment{proof}{\par{\bf Proof}:}{\endproofmark\smallskip}

\textheight=9in
\textwidth=6.5in
\topmargin=-.75in
\oddsidemargin=0.25in
\evensidemargin=0.25in





% -----------------------------------------------------------------
\title{Suites numériques}
\begin{document}
\maketitle


\section*{Exercice 1}

\begin{enumerate}

\item Déterminer $u_n /\; u_{n+1}=4u_n +3$ en fonction de $u_0$
\item Résoudre $u_{n+2}+4u_{n+1}+4u_n=0$
\item Résoudre $u_{n+2}+2u_{n+1}-3u_n=0$
\item Résoudre $u_{n+2}+u_{n+1}+u_n=0$
\end{enumerate}

\section*{Exercice 2}

Démontrer le théorème de Césaro: si $\lim_{n\to\infty} u_n= l$ alors $\lim_{n\to\infty} v_n=\frac{u_0+u_1+\dots+u_n}{n+1}=l$ 

\section*{Exercice 3}

Étudier la suite $(u_n)$ définie par 
$\begin{cases}
u_0=\frac{\pi}{2}\\
u_{n+1}=\sin(u_n)
\end{cases}$

\section*{Exercice 4}

Soit $a>0$, on considère $P_n : x  \mapsto x^n + x^{n-1}+\cdots + x - a$

\begin{enumerate}

\item Montrer que $P_n$ a une seule racine strictement positive, que l'on note $U_n$.
\item Montrer que $(U_n)$ est décroissante.
\item Montrer que $(U_n)$ converge et calculer sa limite en fonction de $a$
\end{enumerate}

\section*{Exercice 5}

Soit $I_n =\int_0^{\frac{\pi}{2}} \sin^n(t)$

\begin{enumerate}
\item Montrer que $I_n =\int_0^{\frac{\pi}{2}} \cos^n(t)$, et $I_n>0$
\item Montrer que $\forall n\in \mathbb{N}$, $I_{n+2}=\frac{n+1}{n+2}I_n$
\item Exprimer $I_n$ à l'aide de factoriels pour $n=2p$ et $n=2p+1$
\item Montrer que $(n+1)I_{n}I_{n+1}=\frac{\pi}{2}$ et $I_{n+2}\leq I_{n+1}\leq I_{n}$
\item Déterminer un équivalent de $I_n$
\end{enumerate}







\end{document}