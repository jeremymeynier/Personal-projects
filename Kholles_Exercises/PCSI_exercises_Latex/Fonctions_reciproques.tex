\documentclass[11pt]{article}
\everymath{\displaystyle}
\usepackage[utf8]{inputenc}  
\usepackage[T1]{fontenc}
\usepackage[francais]{babel}
\usepackage{amsmath,textcomp,amssymb,geometry,graphicx,enumerate}
\usepackage{algorithm} % Boxes/formatting around algorithms
\usepackage[noend]{algpseudocode} % Algorithms
\usepackage{hyperref}
\usepackage{stmaryrd}
\hypersetup{
    colorlinks=true,
    linkcolor=blue,
    filecolor=magenta,      
    urlcolor=blue,
}

\def\Name{Jérémy Meynier}  % Your name
\def\Login{} % nom du chapitre
\def\Homework{N} % Number of Homework
\def\Session{}




\author{\Name \texttt{\Login}}
\markboth{\Session\   \Name}{\Session\  \Name \texttt{\Login}}
\pagestyle{myheadings}
\date{}

\newenvironment{qparts}{\begin{enumerate}[{(}a{)}]}{\end{enumerate}}
\def\endproofmark{$\Box$}
\newenvironment{proof}{\par{\bf Proof}:}{\endproofmark\smallskip}

\textheight=9in
\textwidth=6.5in
\topmargin=-.75in
\oddsidemargin=0.25in
\evensidemargin=0.25in





% -----------------------------------------------------------------
\title{Fonctions réciproques}
\begin{document}
\maketitle

\section*{Exercice 1}

Calculer $\arctan(2) + \arctan(5) + \arctan(8)$

\section*{Exercice 2}

Résoudre $2\arcsin(x)=\arcsin(2x\sqrt{1-x^2})$

\section*{Exercice 3}
\begin{enumerate}
\item Montrer que la fonction $f:\begin{cases}
 [\frac{\pi}{2},\pi[\mapsto[1,+\infty[\\
 \quad\;\; x\mapsto\frac{1}{\sin(x)}
 \end{cases}$ induit une bijection
\item Exprimer sa réciproque
 
\end{enumerate}

\section*{Exercice 4}

Résoudre $\arcsin(2x) = \arcsin(x) + \arcsin(x\sqrt{2})$

\section*{Exercice 5}

Calculer $4\arctan(\frac{1}{5})-\arctan(\frac{1}{239})$

\section*{Exercice 6}

Pour $x\geq0$ et $y\geq0$ montrer que $\arctan(x)-\arctan(y) = \arctan(\frac{x-y}{1+xy})$



\end{document}