\documentclass[11pt]{article}
\everymath{\displaystyle}
\usepackage[utf8]{inputenc}  
\usepackage[T1]{fontenc}
\usepackage[francais]{babel}
\usepackage{amsmath,textcomp,amssymb,geometry,graphicx,enumerate}
\usepackage{algorithm} % Boxes/formatting around algorithms
\usepackage[noend]{algpseudocode} % Algorithms
\usepackage{hyperref}
\usepackage{stmaryrd}
\hypersetup{
    colorlinks=true,
    linkcolor=blue,
    filecolor=magenta,      
    urlcolor=blue,
}

\def\Name{Jérémy Meynier}  % Your name
\def\Login{} % nom du chapitre
\def\Homework{N} % Number of Homework
\def\Session{}




\author{\Name \texttt{\Login}}
\markboth{\Session\   \Name}{\Session\  \Name \texttt{\Login}}
\pagestyle{myheadings}
\date{}

\newenvironment{qparts}{\begin{enumerate}[{(}a{)}]}{\end{enumerate}}
\def\endproofmark{$\Box$}
\newenvironment{proof}{\par{\bf Proof}:}{\endproofmark\smallskip}

\textheight=9in
\textwidth=6.5in
\topmargin=-.75in
\oddsidemargin=0.25in
\evensidemargin=0.25in




% -----------------------------------------------------------------
\title{Matrices}
\begin{document}
\maketitle

\section*{Exercice 1}

Soit $u$ défini par $\forall P\in\mathbb R_3[X],\: u(P) = P' + P$.
\begin{enumerate}
\item Écrire la matrice de $u$ dans la base $\beta=(1,X,X^2,X^3)$.
\item L'endomorphisme est-il inversible? Si oui donner la matrice de $u^{-1}$ dans la base canonique
\end{enumerate} 

\section*{Exercice 2}

Montrer que 2 matrices de $\textit{M}_n(\mathbb{R})$ semblables dans $\textit{M}_n(\mathbb{C})$ sont semblables dans $\textit{M}_n(\mathbb{R})$ 

\section*{Exercice 3}

Soit $A=(a_{ij})\in \textit{M}_n(\mathbb{C})$ tel que $ \forall i\in  \llbracket 1,n \rrbracket, |a_{ij}| > \sum_{j\ne i} |a_{ij}|$. Montrer que $A \in GL_n(\mathbb{C})$

\section*{Exercice 4}

Soit $A \in GL_n(\mathbb{R})$ tel que $A + A^{-1} = I_n.$ Calculer $A^k + A^{-k}$ pour $k\in\mathbb{N}$

\section*{Exercice 5}

Soit $f$ un élément non nul de $\emph{L}(\mathbb{R}^3)$ vérifiant $f^3 + f = 0$. Montrer que $\mathbb{R}^3=Im(f)\oplus Ker(f)$ et que l'on peut trouver une base dans laquelle $f$ a pour matrice $A = \begin{pmatrix} 
0 & 0 & 0 \\
0 & 0 & 1 \\
0 & -1 & 0
\end{pmatrix} $



\section*{Exercice 6}

Soit $A \in M_{3,2}(\mathbb{R})$ et $B \in M_{2,3}(\mathbb{R})$ tel que $AB= \begin{pmatrix}
0 & 0 & 0 \\
0 & 1 & 0 \\
0 & 0 & 1
\end{pmatrix}$. Trouver $BA$

\end{document}
