\documentclass[11pt]{article}
\everymath{\displaystyle}
\usepackage[utf8]{inputenc}  
\usepackage[T1]{fontenc}
\usepackage[francais]{babel}
\usepackage{amsmath,textcomp,amssymb,geometry,graphicx,enumerate}
\usepackage{algorithm} % Boxes/formatting around algorithms
\usepackage[noend]{algpseudocode} % Algorithms
\usepackage{hyperref}
\usepackage{stmaryrd}
\hypersetup{
    colorlinks=true,
    linkcolor=blue,
    filecolor=magenta,      
    urlcolor=blue,
}

\def\Name{Jérémy Meynier}  % Your name
\def\Login{} % nom du chapitre
\def\Homework{N} % Number of Homework
\def\Session{}




\author{\Name \texttt{\Login}}
\markboth{\Session\   \Name}{\Session\  \Name \texttt{\Login}}
\pagestyle{myheadings}
\date{}

\newenvironment{qparts}{\begin{enumerate}[{(}a{)}]}{\end{enumerate}}
\def\endproofmark{$\Box$}
\newenvironment{proof}{\par{\bf Proof}:}{\endproofmark\smallskip}

\textheight=9in
\textwidth=6.5in
\topmargin=-.75in
\oddsidemargin=0.25in
\evensidemargin=0.25in





% -----------------------------------------------------------------
\title{Complexes}
\begin{document}
\maketitle

\section*{Exercice 1}

Prouver à l'aide des complexes que $\cos(p) +\cos(q)=2\cos(\frac{p+q}{2})\cos(\frac{p-q}{2})$

\section*{Exercice 2}

Linéarisation et calcul de $\sum_{k=0}^n \cos(kt)$

\section*{Exercice 3}

Trouver les $z\in\mathbb{C}$ tel que $z^2=3+i$

\section*{Exercice 4}

Trouver les $z/in\mathbb{C}$ tel que $(z+i)^n=(z-i)^n$

\section*{Exercice 5}

Soit $f(z)=\frac{z+1}{z-i}$.

\begin{enumerate}
\item Donner l'ensemble des $z\in\mathbb{C}$ tels que $f(z)\in\mathbb{R}$
\item Donner l'ensemble des $z\in\mathbb{C}$ tels que $|f(z)|=2$
\end{enumerate}

\section*{Exercice 6}

Montrer que $|u+v|^2 +|u-v|^2 = 2(|u|^2 + |v|^2)$ 

\section*{Exercice 7}

Calculer, pour $\alpha\in\mathbb{R}$ avec $\cos(\alpha)\neq 0$, $A_n=\sum_{k=0}^n \frac{\cos(k\alpha)}{\cos^k(\alpha)}$ et $B_n=\sum_{k=0}^n \frac{\sin(k\alpha)}{\cos^k(\alpha)}$

\section*{Exercice 8}

Résoudre $z^2 -(4+2i)z + (11+10i)=0$

\section*{Exercice 9}

Donner une condition nécessaire et suffisante sur $(a,b)\in\mathbb{C}^2$ pour que les solutions de $z^2 +az+b=0$ soient imaginaires pures.



\end{document}